\documentclass[12pt,a4paper]{article}
\usepackage{ctex}
\usepackage{amsmath}
\usepackage{amssymb}
\usepackage{graphicx}
\usepackage{geometry}

\geometry{a4paper,left=2.5cm,right=2.5cm,top=2.5cm,bottom=2.5cm}

\title{数据分析方法}
\author{}
\date{}

\begin{document}
\maketitle

\section{研究方法}

\subsection{数据预处理与基础统计分析}

本研究首先对淄博市情绪数据进行预处理和基础统计分析。数据集包含35,763条记录,每条记录包含地点名称、类型、经纬度、时间段和情绪值等信息。使用Python的pandas库进行数据清洗和基础统计分析,包括描述性统计和分布分析。

\subsection{时空分析方法}

\subsubsection{空间自相关分析}

采用全局Moran's I指数评估情绪值的空间自相关性。计算公式如下:

\begin{equation}
I = \frac{n}{S_0} \frac{\sum_{i=1}^n \sum_{j=1}^n w_{ij}(x_i - \bar{x})(x_j - \bar{x})}{\sum_{i=1}^n (x_i - \bar{x})^2}
\end{equation}

其中:
\begin{itemize}
\item n为观测点数量
\item $w_{ij}$为空间权重矩阵元素
\item $x_i$和$x_j$为位置i和j的情绪值
\item $\bar{x}$为情绪值平均值
\item $S_0$为所有空间权重的总和
\end{itemize}

使用DistanceBand方法构建基于距离的空间权重矩阵,阈值设置为0.01,采用行标准化处理。

\subsubsection{时间序列分析}

对时间序列数据进行季节性分解,使用加法模型:

\begin{equation}
Y_t = T_t + S_t + R_t
\end{equation}

其中:
\begin{itemize}
\item $Y_t$为观测值
\item $T_t$为趋势项
\item $S_t$为季节项
\item $R_t$为残差项
\end{itemize}

使用statsmodels库的seasonal\_decompose函数进行分解,周期设置为7天。

\subsection{聚类分析}

采用K-means聚类算法对情绪数据进行空间聚类分析。特征包括经度、纬度和情绪值,进行如下步骤:

\begin{enumerate}
\item 数据标准化:使用StandardScaler进行特征标准化
\item 最优聚类数确定:使用肘部法则(Elbow Method)
\item 聚类实现:采用KMeans算法,随机种子设为42
\end{enumerate}

聚类目标函数:

\begin{equation}
J = \sum_{j=1}^k \sum_{i=1}^n \|x_i^{(j)} - c_j\|^2
\end{equation}

其中:
\begin{itemize}
\item k为聚类数
\item n为样本数
\item $x_i^{(j)}$为第j类中的第i个样本
\item $c_j$为第j类的聚类中心
\end{itemize}

\subsection{统计检验方法}

\subsubsection{方差分析}

对不同场所类型的情绪差异进行单因素方差分析(One-way ANOVA),并使用Tukey HSD进行事后检验。

\subsubsection{非参数检验}

使用Kruskal-Wallis H检验分析不同时间段的情绪差异,适用于可能不满足正态分布假设的数据。

\subsubsection{置信区间估计}

采用t分布计算情绪值的95\%置信区间:

\begin{equation}
CI = \bar{x} \pm t_{(n-1, \alpha/2)} \times \frac{s}{\sqrt{n}}
\end{equation}

其中:
\begin{itemize}
\item $\bar{x}$为样本均值
\item s为样本标准差
\item n为样本量
\item $t_{(n-1, \alpha/2)}$为自由度为n-1的t分布的临界值
\end{itemize}

\subsection{可视化方法}

采用多种可视化技术展示分析结果:
\begin{enumerate}
\item 使用folium库生成交互式热力图
\item 使用matplotlib和seaborn库绘制统计图表
\item 使用箱线图展示不同类别的情绪分布
\item 使用时序图展示情绪随时间的变化趋势
\end{enumerate}

所有分析使用Python 3.8及其科学计算生态系统实现,主要依赖库包括:pandas、numpy、scipy、sklearn、statsmodels、libpysal、esda等。分析结果以图表和数值统计形式保存在analysis\_results目录下。

\end{document} 